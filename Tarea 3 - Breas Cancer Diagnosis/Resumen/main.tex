\documentclass[11pt,letterpaper]{article}
\usepackage[activeacute,spanish]{babel}
%\usepackage[ansinew]{inputenc}
\usepackage[utf8]{inputenc}
% \usepackage[latin1]{inputenc}
\usepackage[letterpaper,includeheadfoot, top=0.5cm, bottom=3.0cm, right=2.0cm, left=2.0cm]{geometry}
\renewcommand{\familydefault}{\sfdefault}

\usepackage{graphicx}
\usepackage{color}
\usepackage{hyperref}
\usepackage{amssymb}
\usepackage{url}
%\usepackage{pdfpages}
\usepackage{fancyhdr}
\usepackage{hyperref}
\usepackage{subfig}
\usepackage{colortbl}

\usepackage{listings} %Codigo
\lstset{language=C, tabsize=4,framexleftmargin=5mm,breaklines=true}
%%%%%%%%%%%%%%%%%%%%%%%%%%%%%%%%%%%%%%%%%%%%%
% PARCHE PARA PERMIRIR UTILIZAR BIBLATEX EN ESTA PANTLLA
%\PassOptionsToPackage{square,numbers}{natbib}
%\RequirePackage{natbib}  
%%%%%%%%%%%%%%%%%%%%%%%%%%%%%%%%%%%%%%%%%%%%%

\usepackage[backend=bibtex,sorting=none]{biblatex}
% Estas lineas permiten romper los hipervinculos muy largos !!!!
\setcounter{biburllcpenalty}{7000}
\setcounter{biburlucpenalty}{8000}
\addbibresource{references.bib}

% Actualiza en automático la fecha de las citas de internet a la fecha de la compilación del documento
\usepackage{datetime}
\newdateformat{specialdate}{\twodigit{\THEDAY}-\twodigit{\THEMONTH}-\THEYEAR}
\date{\specialdate\today}

\begin{document}
%\begin{sf}
% --------------- ---------PORTADA --------------------------------------------
\newpage
\pagestyle{fancy}
\fancyhf{}
%------------------ TÍTULO -----------------------
\vspace*{6cm}
\begin{center}
\Huge  {Breas Cancer Diagnostic and Prognosis Via Linear Programing}
\vspace{1cm}
\end{center}
%----------------- NOMBRES ------------------------
\begin{center}
\begin{tabular}{ll}
Autor: & José Luis Pérez Avila\\
& Cd. Victoria, Tamaulipas.
\end{tabular}
\end{center}
%\date{}

\section{Definicion del Problema}
El \textbf{obejtivo} que desarrolla este artículo es el de usar una técnica de machine learning basado en pogramación linear para incrementar la precision y objetividad del diagnostico y que tan avanzado se encuentra el cancer de mama. 

Las \textbf{preguntas} que se desarrollaron fueron, 1. ¿Cuantas mujeres en Estados Unidos han sido diagnosticadas en 1987 y cuantas moriran por ello?, ¿Es posible que una mujer sobreviva al cancer de mama si se le da tratamiento en una fase tembrana de este?.

Su \textbf{justificación} es que el implementar el tratamiento correcto a una mujer con cancer de mama en una etapa temprana influencia mucho en pronostico a largo plazo.

La \textbf{viabilidad} es utilizar dos aplicaciones significativas de programación lineal en el campo de investigación del cancer de mama, un programa de computadora llamado Xcyt y un dataset entrenado con 569 pacientes.

\textbf{Concecuencias} del estudio, al encontrar una forma de diagnosticar el cancer de mama mas efectiva sería de gran utilidad para el área de medicina detectando el cancer de manera precisa y en que etapa de encuentra.

La \textbf{hipotesis} es implementar una técnica de machine learning basado en programación linear para incrementar la precisión, objetividad del diagnostico y que tan avanzado se encuentra el cancer de mamá en Estados unidos y reducir la tasa de mortalidad por esta enfermedad. 

\end{document}













